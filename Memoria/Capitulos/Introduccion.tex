\chapter{Introducción}
\label{cap:introduccion}

\chapterquote{¿Alguna vez has retirado un humano por error?}{Rachael - Blade Runner (1982)}


En este capítulo se presenta la motivación que incentiva el desarrollo de este trabajo (véase la sección \ref{motivacion}) y los objetivos que se pretenden alcanzar (apartado \ref{obj})

\section{Motivación}\label{motivacion}
El reciente desarrollo y el consecuente auge que los \textit{smartphones} han llevado a cabo en las últimas décadas, ha traído consigo innumerables cambios de paradigma, tanto en el término de lo tecnológico como en el ámbito social. Precisamente, el fácil y cada vez más sencillo acceso que supuso la aparición de los smartphones conectados a una red 3G a principios de los 2000, permitió acercar a un público emergente y cada vez mayor, herramientas, hoy en día consideradas esenciales, como las videoconferencias o la mensajería instantánea. Fue precisamente este sistema de mensajería ``pionero'' el que terminó evolucionando y desembocando en el desarrollo de distintas aplicaciones que han ayudado a la popularización de la comunicación en tiempo real, llegando a enviar hasta cien mil millones mensajes de manera diaria en 2020. Aunque esta serie de aplicaciones cuentan con la principal característica de ser más cercanas, informales e inmediatas –de modo que la necesidad de inmediatez y la urgencia de los emisores queda satisfecha prácticamente al segundo–, esto no les ha sido suficiente para relevar totalmente de sus tareas a sistemas de mensajería más ``veteranos'' como el correo electrónico, que requieren de un mayor tiempo de dedicación para su correcta utilización, dada la mayor complejidad, riqueza y elaboración del lenguaje que en ellos conviene presentar, debido a su carácter y ``personalidad'' más formal. Si bien es cierto que, aunque hoy en día, general y habitualmente el correo electrónico se destine a ámbitos más formales o institucionales, como exordio de una comunicación que probablemente evolucione hacia otro tipo de plataformas o, más informalmente, para el envío de archivos, este continúa siendo un medio de comunicación de uso habitual; en el caso de algunas personas, este uso se extiende incluso al ámbito diario, pero con la clara y evidente desventaja de que requiere una mayor atención por parte del usuario.

Resulta evidente que existe una clara brecha entre aplicaciones de mensajería instantánea y correo electrónico y que esta, debe ser solventada para facilitar la experiencia y el uso del correo a sus usuarios; por ello, se propone IRIS. El sistema IRIS (\textit{Intelligent Response Instantly Sent}) se postula como un asistente virtual cuyo objetivo es mejorar la experiencia de usuario en la utilización del correo electrónico, abordando la problemática de la dedicación requerida que este exige; en consecuencia, el \textit{modus operandi} de IRIS es sencillo. Al utilizar IRIS, el usuario postula la idea que quiere transmitir y el tema en torno al que debe girar el cuerpo del mensaje y esto es recibida por el asistente, que mediante técnicas de generación de lenguaje natural, tratará de expresar dicha idea en un formato de texto de manera extensa, para posteriormente enviar el correo electrónico a su destinatario.


\section{Objetivos}\label{obj}
Los principales objetivos de este trabajo son el estudio, la implementación y la evaluación de las distintas ténicas de generación de lenguaje natural (NLG) que permiten redactar correos electrónicos de manera automática a partir de pequeñas representaciones semánticas. Para lograrlo, se entrará en detalle tanto de las distintas arquitecturas del campo de la NLG y se profundizará en las alternativas para producir estas representaciones semánticas que luego puedan dar lugar a mensajes completos. En concreto, se estudiarán los llamados Information Items, utilizados en el campo del resumen automático de textos, para implementar estas representaciones semánticas y evaluarlas con las dos arquitecturas principales hoy en día en el campo de la NLG: los modelos transformers y realizers.
