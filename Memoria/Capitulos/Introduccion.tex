\chapter{Introducción}
\label{cap:introduccion}

\chapterquote{¿Alguna vez has retirado un humano por error?}{Rachael - Blade Runner (1982)}


En este capítulo se presenta la motivación que incentiva el desarrollo de este trabajo (véase la sección \ref{motivacion}), los objetivos que se pretenden alcanzar (apartado \ref{obj}) y la estructura del documento completo (sección \ref{estructura}).

\section{Motivación}\label{motivacion}
El reciente desarrollo y el consecuente auge que los \textit{smartphones} han llevado a cabo en las últimas décadas, ha traído consigo innumerables cambios de paradigma, tanto en el término de lo tecnológico como en el ámbito social. Precisamente, el fácil y cada vez más sencillo acceso que supuso la aparición de los smartphones conectados a una red 3G a principios de los 2000, permitió acercar a un público emergente y cada vez mayor, herramientas, hoy en día consideradas esenciales, como las videoconferencias o la mensajería instantánea. Fue precisamente este sistema de mensajería ``pionero'' el que terminó evolucionando y desembocando en el desarrollo de distintas aplicaciones que han ayudado a la popularización de la comunicación en tiempo real, llegando a enviar hasta cien mil millones mensajes de manera diaria en 2020. Aunque esta serie de aplicaciones cuentan con la principal característica de ser más cercanas, informales e inmediatas –de modo que la necesidad de inmediatez y la urgencia de los emisores queda satisfecha prácticamente al segundo–, esto no les ha sido suficiente para relevar totalmente de sus tareas a sistemas de mensajería más ``veteranos'' como el correo electrónico, que requieren de un mayor tiempo de dedicación para su correcta utilización, dada la mayor complejidad, riqueza y elaboración del lenguaje que en ellos conviene presentar, debido a su carácter y ``personalidad'' más formal. Si bien es cierto que, aunque hoy en día, general y habitualmente el correo electrónico se destine a ámbitos más formales o institucionales, como exordio de una comunicación que probablemente evolucione hacia otro tipo de plataformas o, más informalmente, para el envío de archivos, este continúa siendo un medio de comunicación de uso habitual; en el caso de algunas personas, este uso se extiende incluso al ámbito diario, pero con la clara y evidente desventaja de que requiere una mayor atención por parte del usuario.

Resulta evidente que existe una clara brecha entre aplicaciones de mensajería instantánea y correo electrónico y que esta, debe ser solventada para facilitar la experiencia y el uso del correo a sus usuarios; por ello, se propone IRIS. El sistema IRIS (\textit{Intelligent Response Instantly Sent}) se postula como un asistente virtual cuyo objetivo es mejorar la experiencia de usuario en la utilización del correo electrónico, abordando la problemática de la dedicación requerida que este exige; en consecuencia, el \textit{modus operandi} de IRIS es sencillo. Al utilizar IRIS, el usuario postula la idea que quiere transmitir y el tema en torno al que debe girar el cuerpo del mensaje y esto es recibida por el asistente, que mediante técnicas de generación de lenguaje natural, tratará de expresar dicha idea en un formato de texto de manera extensa, para posteriormente enviar el correo electrónico a su destinatario.


\section{Objetivos}\label{obj}
Los principales objetivos de este trabajo son el estudio, la implementación y la evaluación de las distintas ténicas de generación de lenguaje natural (NLG) que permiten redactar correos electrónicos de manera automática a partir de pequeñas representaciones semánticas. Para lograrlo, se entrará en detalle tanto de las distintas arquitecturas del campo de la NLG y se profundizará en las alternativas para producir estas representaciones semánticas que luego puedan dar lugar a mensajes completos. En concreto, se estudiarán los llamados Information Items, utilizados en el campo del resumen automático de textos, para implementar estas representaciones semánticas y evaluarlas con las dos arquitecturas principales hoy en día en el campo de la NLG: los modelos transformers y realizers.

Se pretende desarrollar soluciones capaces de, recibiendo como entrada los Information Items, redacten un correo electrónico por completo como lo haría el usuario. Para lograrlo, se elegirá un corpus suficientemente grande de mensajes que permita entrenar el modelo adecuadamente. Este conjunto de documentos, se analizará para conocer las características y propiedades que lo definen, se llevarán a cabo tareas de limpieza de los cuerpos de los correos electrónicos y de procesamiento  y almacenamiento de los mismos. Para esta última fase, se definirá una estructura adecuada con un sistema de almacenamiento óptimo que permita gestionar la información de manera eficiente.

Una vez se cuente con un sistema de almacenamiento y se hayan llevado a cabo las tareas de análisis y procesamiento de los datos, se implementará el modelo de la arquitectura de generación de lenguaje natural que mejor se ajuste al problema propuesto y se entrenará con los datos del corpus. Finalmente, se pretende extraer resultados que den a conocer el rendimiento y performance del sistema desarrollado, y así poder evaluar la eficacia de la solución propuesta ante el problema de redacción automática de correos electrónico con técnicas de generación de lenguaje natural.

\section{Estructura del documento}\label{estructura}
Este documento presenta los detalles del trabajo desarrollado, mostrando la justificación y motivación para las diferentes decisiones y pasos que se han tomando, y deja patente los distintos resultados, conclusiones y razonamientos a los que se ha llegado tras analizar todo lo implementado.

Comienza, como se ha observado, con una introducción (capítulo \ref{cap:introduccion}) que pretende dar una idea del problema al que se intenta dar respuesta y la utilidad de la solución que se construye. A esta motivación (sección \ref{motivacion}) le sigue la presentación de los objetivos (apartado \ref{obj}) que se abarcarán a lo largo de todo el proyecto. Estos propósitos serán los que dirigirán las decisiones tomadas en todo el trabajo y hacia los que se pondrá el foco de atención en todo momento. La introducción acaba con esta sección en la que se explica la estructura del documento.

Una vez se ha introducida adecuadamente la motivación y los objetivos del estudio, se lleva a cabo una revisión general del estado de la cuestión (capítulo \ref{cap:estadoDeLaCuestion}) de los temas que abarca el proyecto. Con este capítulo, el lector podrá adoptar los conocimientos necesarios que constituyen la base práctica y teórica del desarrollo presentado. Comenzando con los fundamentos del correo electrónico (sección \ref{s:email}), se brinda una visión del panorama general de cómo este medio de comunicación influye notablemente en la sociedad actual. Además, también se dan las especificaciones técnicas necesarias para entender el formato de los e-mails (el protocolo MIME explicado en el apartado \ref{ss:mime}), cómo se envían (el protocolo SMTP queda reflejado en la sección \ref{ss:smtp}) y cómo se reciben (protocolos POP e IMAP que se presentan en las secciones \ref{ss:pop} y \ref{ss:imap}, respectivamente).

En el capítulo \ref{cap:estadoDeLaCuestion}, además de entrar en detalle acerca del correo electrónico, se ofrece los conceptos básicos necesario de la generación de lenguaje natural (apartado \ref{s:nlg}): en qué consiste esta rama de la inteligencia artificial y sus aplicaciones más comunes (sección \ref{ss:quenlg}), las principales arquitecturas más utilizadas en los problemas de NLG (apartado \ref{ss:arqnlg}) y una breve introducción al ámbito del resumen abstractivo de textos (sección \ref{ss:resumen}) del cual se utilizarán técnicas para abordar el problema que compete al proyecto.

A continuación, cuando ya se cuenta con los conocimientos requeridos para enfrentarse al problema, en el capítulo \ref{cap:descripcionTrabajo}, se presenta todo el trabajo realizado. Primero se introducen brevemente las tecnologías que se manejan en la fase de desarrollo del proyecto (apartado \ref{s:tech}): spaCy (sección \ref{ss:spacy}), textaCy (apartado \ref{ss:textacy}), tensorflow (punto \ref{ss:tf}) y MongoDB (sección \ref{ss:mongodb}). Se continúa reflejando todos los pasos que engloban el análisis de datos preliminar del corpus escogido (sección \ref{s:analisis}) y se finaliza explicando los problemas y detalles de las implementaciones de la solución utilizando la arquitectura realizer (apartado \ref{s:realizer}), transformer (punto \ref{s:transformer}) y los resultados obtenidos con el desarrollo (sección \ref{s:resultados}).

Por último, en el capítulo \ref{cap:conclusiones}, se sacan a relucir las conclusiones que pueden extraerse de este estudio (apartado \ref{s:concl}) y las distintas vías de trabajo abiertas que deja este proyecto sobre las que se podría continuar con él (sección \ref{s:fut}).