\chapter{Conclusiones y Trabajo Futuro}
\label{cap:conclusiones}

\chapterquote{Difícil de ver es. Siempre en movimiento el futuro está}{Yoda - Star Wars: Episodio III - La venganza de los Sith (2005)}

Tras el desarrollo de este trabajo, en este capítulo se presentan las conclusiones que pueden extraerse de este estudio (se explican en la sección \ref{s:concl}). Justo después, las posibles opciones para la continuación de este trabajo quedan expuestas en la sección \ref{s:fut}, con el fin de continuar con el estudio de la generación de lenguaje natural a partir de representaciones semánticas.

\section{Conclusiones}\label{s:concl}

Hoy en día, el correo electrónico es un sistema de comunicación que se utiliza tanto en el ámbito profesional como en el personal. A través de él se establecen conversaciones sobre el trabajo, los estudios, relaciones íntimas, etcétera. Sin embargo, el gran número de e-mails que se reciben y envían cada día, comienza a obligar a sus usuarios a dedicar una notable cantidad de tiempo a atender su bandeja de entrada y pensar cuál es la mejor forma de redactar los mensajes, para transmitir la idea que se tiene en mente. Pero, ¿y si se pudiera escribir de forma automática con tan solo introducir como entrada dicha idea o concepto sobre el que giraría el cuerpo del mensaje? Este problema puede ser resuelto mediante el uso de técnicas de generación de lenguaje natural, un campo de la inteligencia artificial que se enfrenta al reto de producir textos imitando la forma en que los humanos se comunican entre sí. Dentro de esta rama, destacan dos tipos de arquitecturas: los modelos realizer y los modelos transformer. La primera aproximación consiste en un pipeline de tareas que, poco a poco, construyen el texto de salida; mientras que la segunda hace uso de los modelos de atención y las arquitecturas codificador-decodificador de deep learning. En este trabajo se trata de evaluar la viabilidad de implementar cada una de ellas para enfrentarse al problema de redacción automática de correos electrónicos, desarrollar la solución y valorar los resultados obtenidos.

Para representar esa idea que el usuario posee, se hace uso de los llamados Information Items, entidades que almacenan la mínima información semántica. Este elemento proviene de la rama del resumen automático de textos. Los Information Items serán la entrada del sistema desarrollado, representando ese concepto que posee el usuario acerca de lo que quiere redactar en el correo electrónico. Concretamente, se implementan mediante tuplas sujeto-verbo-objeto. Sin embargo, como se ha mostrado en el trabajo, esta definición no funciona adecuadamente en ninguna de las dos arquitecturas, ya que obliga al sistema de generación de lenguaje natural a producir construcciones lingüísticas con elementos semánticos no transmitidos por el usuario y que no puede obtener de otra manera.

En lo que se refiere al modelo realizer, las tuplas sujeto-verbo-objeto, solventan el problema de construcción de estructuras ad hoc para hacer frente a la fase de determinación del contenido, es decir, permiten implementar dicha tarea a pesar de que los correos electrónicos no se enmarquen en uno o varios dominios específico. No obstante, como en esta arquitectura el resto de fases poseen una alta dependencia de la salida de la determinación del contenido, que serían los Information Items, el texto producido se restringe a cada uno de ellos sin poder aportar más información semántica. Esto se debe a que no se cuenta con una base de conocimiento o módulo de razonamiento general de los que sacar dicha información extra que se pueda añadir al texto producido.

Respecto al modelo transformer, podría ocurrir que se añadiera información extra que aumentara el tamaño y la riqueza del cuerpo del correo electrónico. Sin embargo, al no poseer más contexto que los Information Items, si se incluyera, se trataría de estructuras lingüísticas basadas en la frecuencia de aparición del resto de correos electrónicos. Esto, limita enormemente la salida del sistema, así que requiere una inmensa cantidad de documentos en el corpus, mayor de la que se tiene, para poder producir el texto como esperaría el usuario final. Todas estas razones justifican los resultados obtenidos con dicha arquitectura y demuestran que la aproximación con esta definición de las representaciones semánticas que constituyen los Information Items, no permite alcanzar una generación de lenguaje natural de buena calidad, coherencia y cohesión.

\section{Trabajo futuro}\label{s:fut}
En vista de los resultados obtenidos, la principal vía de trabajo futuro es el estudio de la redacción automática de correos electrónicos utilizando otras implementaciones más complejas de los Information Items. La clave reside en que almacenen la suficiente información como para ser capaces de generar por completo el mensaje, pero no excesiva como para que el usuario tenga que redactar prácticamente todo el texto. Varias alternativas se han propuesta en el capítulo \ref{cap:estadoDeLaCuestion}, que pueden ser estudiadas individualmente o en conjunto. Por ejemplo, sería posible combinar el etiquetado del rol semántico con técnicas de desambiguación del sentido de las palabras.

No obstante, no se debe descartar la posibilidad de que quizás, con un mayor conjunto de entrenamiento, la propuesta desarrollada a lo largo de este trabajo obtenga resultados satisfactorios. Por esa razón, otra opción para continuar el trabajo desarrollado es la reutilización de los módulos implementados (que pueden encontrarse en el repositorio correspondiente\footnote{\url{https://github.com/carlosmmorera/NLG-AI-Master-Thesis}}) y entrenarlos con un corpus mayor que permite exprimir la utilidad de los InIts.

Por otro lado, es posible estudiar mejoras de cara a la arquitectura realizer con las que quizás, se podrían obtener resultados más satisfactorios. Estas mejoras van desde la reformulación de los métodos de redacción de pequeñas partes del mensaje, como la implementación de un modelo probabilístico para la generación automática del saludo en los correos electrónicos, hasta el desarrollo de un sistema de razonamiento con el que deducir otros Information Items e incluirlos al cuerpo del mensaje.

En cuanto a trabajo futuro en la arquitectura transformer, es razonable pensar que la división del problema de redacción en distintas partes, como lo que se ha propuesto antes de redactar fragmentos del e-mail, pueda concluir en resultados más satisfactorios. Además, también podría estudiarse la posibilidad de incluir más información de entrada, como los metadatos del correo electrónico (destinatario, asunto, etcétera) para extraer de ellos información extra que se pueda incluir en la redacción.

Estas son algunas vías posibles de trabajo futuro, de cara al estudio del caso de uso de redacción automática de correos electrónicos utilizando técnicas avanzadas de inteligencia artificial y, en concreto, del campo de la generación de lenguaje natural.