\section{Enron corpus}
Para llevar a cabo este trabajo, se ha elegido el corpus conocido como Enron\footnote{\url{http://www-2.cs.cmu.edu/~enron/}}, dado que los correos electrónicos que contiene pertenecieron a la empresa con el mismo nombre. Precisamente se hicieron públicos tras una investigación legal llevada a cabo a esta compañía por parte de la Comisión Federal de Regulación de la Energía\footnote{\url{https://www.ferc.gov/}} de Estados Unidos.

Enron corpus contiene 517.401 correos electrónicos de 150 usuarios distintos. Además de la ventaja de la gran cantidad de elementos pertenecientes a este dataset, también ha sido elegido por encontrar diversos trabajos sobre este mismo conjunto de e-mails, como el llevado a cabo por \cite{klimt2004introducing}.