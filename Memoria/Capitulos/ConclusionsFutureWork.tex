\chapter{Conclusions and Future Work}
\label{cap:conclusions}

\chapterquote{Difficult to see. Always in motion the future is}{Yoda - Star Wars: Episode III - Revenge of the Sith (2005)}

After the development of this work, this chapter presents the conclusions that can be drawn from this study (they are explained in section \ref{s:concle}). Right after, the possible options for the continuation of this work are exposed in section \ref{s:fute}, in order to continue with the study of the use case of automatic e-mail composition from semantic representations using advanced artificial intelligence techniques, in particular, from the field of natural language generation .

\section{Conclusions}\label{s:concle}

Nowadays, e-mail is a communication system used in both professional and personal environments. Through it, conversations about work, studies, intimate relationships, and so on are established. However, the large number of e-mails received and sent every day is beginning to force users to spend a considerable amount of time attending to their inbox and thinking about the best way to write messages in order to convey the idea they have in mind. But what if you could write automatically by simply entering as input the idea or concept around which the body of the message would revolve? This problem can be solved by using natural language generation techniques, a field of artificial intelligence that faces the challenge of producing texts imitating the way humans communicate with each other. Within this branch, two types of architectures stand out: realizer models and transformer models. The first approach consists of a pipeline of tasks that, little by little, build the output text; while the second one makes use of attention models and deep learning encoder-decoder architectures. In this work we try to evaluate the feasibility of implementing each of them to face the problem of automatic e-mail composition, develop the solution and evaluate the results obtained.

To represent the idea that the user has, we make use of the so-called Information Items, entities that store the minimum semantic information. This element comes from the branch of automatic text summarization. The Information Items will be the input of the developed system, representing the concept that the user has about what he wants to write in the e-mail. Concretely, they are implemented by means of subject-verb-object tuples. However, as it has been shown in the paper, this definition does not work properly in any of the two architectures, since it forces the natural language generation system to produce linguistic constructions with semantic elements not transmitted by the user and that cannot be obtained in any other way.

As far as the realizer model is concerned, subject-verb-object tuples solve the problem of constructing ad hoc structures to cope with the content determination phase, i.e. they allow the implementation of this task even though the e-mails do not fall into one or more specific domains. However, as in this architecture the rest of the phases are highly dependent on the output of the content determination, which would be the Information Items, the text produced is restricted to each of them without being able to provide more semantic information. This is because there is no knowledge base or general reasoning module from which to draw such extra useful information that can be added to the text produced.

With respect to the transformer model, it could happen that extra information could be added to increase the size and richness of the body of the e-mail. However, as it has no context other than Information Items, if it were included, it would be linguistic structures based on the frequency of appearance of the rest of the e-mails. This greatly limits the output of the system, so it requires an immense amount of documents in the corpus, larger than what is available, to be able to produce the text as the end user would expect. All these reasons justify the results obtained with such an architecture and demonstrate that the approach with this definition of the semantic representations that constitute the Information Items, does not allow to achieve a natural language generation of good quality, coherence and cohesion.

\section{Future Work}\label{s:fute}
In view of the results obtained, the main avenue for future work is the study of automatic e-mail composition using other more complex implementations of Information Items. The key is that they store enough information to be able to generate the entire message, but not so much that the user has to compose almost all the text. Several alternatives have been proposed in the chapter, which can be studied individually or together. For example, it would be possible to combine semantic role labeling with word sense disambiguation techniques.

However, one should not rule out the possibility that perhaps, with a larger training set, the approach developed throughout this work might obtain satisfactory results. For that reason, another option to continue the work developed is the reuse of the implemented modules (which can be found in the corresponding repository) and train them with a larger corpus that allows to squeeze the usefulness of the InIts.

On the other hand, it is possible to study improvements to the realizer architecture with which, perhaps, more satisfactory results could be obtained. These improvements range from the reformulation of the methods of writing small parts of the message, such as the implementation of a probabilistic model for the automatic generation of the greeting in e-mails, to the development of a reasoning system with which to deduce other Information Items and include them in the body of the message.

As for future work on the transformer architecture, it is reasonable to think that the division of the writing problem into different parts, such as what has been proposed before to write fragments of the e-mail, may lead to more satisfactory results. In addition, the possibility of including more input information, such as e-mail metadata (recipient, subject, etc.) could also be explored in order to extract extra information from them that can be included in the redaction.

These are some possible avenues for future work, with a view to studying the use case of automatic e-mail composition using advanced artificial intelligence techniques and, in particular, from the field of natural language generation.


