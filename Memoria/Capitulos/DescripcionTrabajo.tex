\chapter{Descripción del Trabajo}
\label{cap:descripcionTrabajo}

\chapterquote{¡Datos! ¡Datos! ¡Datos! - exclamó con impaciencia - No puedo hacer ladrillos sin arcilla}{El misterio de Copper Beeches\\Arthur Conan Doyle (1892)}

Tras comprender los conceptos necesarios para hacer frente al problema de construir un sistema capaz de redactar correos electrónicos a partir de pequeñas representaciones semánticas, se puede proceder a la explicación de lo desarrollado en este trabajo. En primer lugar es introducen brevemente las tecnologías utilizadas (véase la sección \ref{s:tech}). A continuación se realiza un análisis de datos previo para conocer las propiedades y detalles de los datos con los que se va a trabajar y entrenar el modelo (consultar el apartado \ref{s:analisis}). Después se entra en detalle acerca de la arquitectura elegida, razones por las que no se han conseguido resultados con la arquitectura descartada y los ajustes concretos que se han implementado para abordar el problema que compete a este trabajo (secciones \ref{s:realizer} y \ref{s:transformer}). Por último, se presentan los resultados obtenidos y una reflexión acerca de a qué se deben (léase el apartado \ref{s:resultados}).

\section{Tecnologías utilizadas}\label{s:tech}
Antes de comenzar con el desarrollo del trabajo realizado, en esta sección se introducen las distintas tecnologías que han sido empleadas para la implementación del proyecto. Dado que el estudio gira en torno a los cuerpos de los correos electrónicos, es imprescindible un analizador sintáctico que facilite el estudio del corpus de partida. Para cubrir esta necesidad, se ha elegido spaCy, herramienta que se detalla en la sección \ref{ss:spacy}. Por otro lado, en lo relativo a las tareas relacionadas con el procesamiento de lenguaje natural, también se necesita poder obtener los Information Items dado un texto de entrada. Este problema puede implementarse gracias a la librería textaCy (véase el apartado \ref{ss:textacy}).

En cuanto al desarrollo de la arquitectura transformer, la cual consta de numerosas capas de redes neuronales, como se explicó en la sección \ref{sss:transformer}, será de gran ayuda la librería desarrollada por Google llamada Tensorflow (consúltese el apartado \ref{ss:tf}). Gracias a ella será posible presentar un modelo de aprendizaje automático que aborde todos los problemas de la generación de lenguaje natural.

Por último, se introduce el sistema de almacenamiento con el que se ha trabajado para contar con una gestión eficiente de los datos que se manejan: MongoDB (sección \ref{ss:mongodb}).

\subsection{spaCy}\label{ss:spacy}
Para ser capaces de llegar a conclusiones y estudiar el corpus de correos electrónicos seleccionado (véase la sección \ref{ss:enron}), se debe poder analizar el cuerpo de los mismos. Es decir, se necesita un analizador sintáctico que permita separar los diferentes textos en tokens (o dicho de otro modo, segmentar el texto en palabras, signos de puntuación, etcétera) y obtener diferentes características de ellos (como su categoría gramatical). Para conseguir ese objetivo, se va a utilizar la librería spaCy\footnote{\url{https://spacy.io/}}.

En esta sección se explican las razones por las que se elige spaCy (véase la sección \ref{ssect:spacywhy}) y su utilidad en el proyecto (véase la sección \ref{ssect:spacyut}).

\subsubsection{spaCy frente a otros analizadores sintácticos}\label{ssect:spacywhy}

Se ha elegido spaCy como analizador sintáctico frente a otros por varias razones, apoyadas por investigaciones publicadas como la realizada por \cite{choi2015depends}, y que se explican a continuación.

\begin{figure}[h]
	\centering%
	\includegraphics[width = 0.75\textwidth]{Imagenes/Bitmap/spacyeval.png}%
	\caption{Benchmarks de los distintos analizadores sintáctos}%
	Imagen extraída de \url{https://spacy.io/usage/facts-figures#benchmarks}
	\label{fig:spacyeval}
\end{figure}

Una evaluación publicada por \textit{Yahoo! Labs} y la Universidad Emory, como parte de un estudio de las tecnologías de análisis sintáctico actuales \citep{choi2015depends}, observó que ``spaCy es el analizador sintáctico voraz más rápido'' y su precisión está dentro del 1\% de los mejores existentes (como podemos ver en la figura \ref{fig:spacyeval}). Los pocos sistemas que son más precisos son, al menos, 20 veces más lentos. La velocidad es un factor importante cuando se quiere implementar sistemas complejos que se enfrentan a textos largos o a un gran número de documentos (como es el caso de este trabajo, en el que se quieren analizar todos los correos electrónicos posibles).

\begin{figure}[h]
	\centering%
	\includegraphics[width = 0.85\textwidth]{Imagenes/Bitmap/spacyspeed.png}%
	\caption{Tiempo de procesamiento por documento de varias librerías de NLP}%
	Imagen extraída de \url{https://spacy.io/usage/facts-figures#benchmarks}
	\label{fig:spacyspeed}
\end{figure}

Los resultados de \cite{choi2015depends} y las discusiones posteriores ayudaron a spaCy a desarrollar una novedosa técnica para mejorar la precisión del sistema, la cual fue publicada en un trabajo conjunto con la Universidad de Macquarie \citep{honnibal2015improved}. Por este motivo, se ha elegido una versión de spaCy que aprovecha esta técnica.

Además, no sólo en general, sino en cada tarea particular (tokenización, etiquetado y análisis sintáctico), spaCy es la más rápida si la comparamos con otras librerías de procesamiento del lenguaje natural. Esto se muestra en la figura \ref{fig:spacyspeed}, donde se puede observar tanto los tiempos absolutos (en milisegundos) como el rendimiento relativo (normalizado a spaCy). Los sistemas que tienen valores más bajos son más rápidos en sus tareas.

\subsubsection{Utilidades de spaCy}\label{ssect:spacyut}
Se puede definir spaCy como una biblioteca de procesamiento de lenguaje natural de Python diseñada específicamente para ser una biblioteca útil para implementar sistemas listos para la producción. Por esta razón, tiene una gran cantidad de utilidades diferentes. Sin embargo, sólo se necesitarán las que realizan el \textit{Tokenizer}.

La clase \textit{Tokenizer} de spaCy se encarga de dividir el mensaje dado en las diferentes palabras que lo constituyen y obtener varias características sobre ellas. Interesan los atributos que se pueden observar en la tabla \ref{tab:attspacy}. Además de su categoría gramatical, da más información (que no nos interesa) en función de su categoría léxica, como su género, número, tiempo verbal o, incluso, el tipo de adverbio.

\begin{table}[h]
	\begin{tabular}{|l|l|p{0.675\linewidth}|}
		\hline
		\textbf{Atributo} & \textbf{Tipo} & \textbf{Explicación}                                                                     \\ \hline
		is\_punct          & bool          & Indica si el token es un signo de puntuación \\ \hline
		is\_right\_punct   & bool          & Indica si el token es un signo de puntuación derecho (como el paréntesis cerrado). \\ \hline
		is\_left\_punct    & bool          & Indica si el token es un signo de puntuación izquierdo\\ \hline
		is\_bracket        & bool          & Indica si el token es un paréntesis\\ \hline
		like\_url          & bool          & Indica si el token es una url \\ \hline
		like\_email        & bool          & Indica si el token es una dirección de correo electrónico\\ \hline
		lema\_             & string        & Forma base del token sin sufijos o inflexiones\\ \hline
		is\_stop           & bool          & Indica si el token es una stop word\\ \hline
		pos\_              & string        & Categoría gramatical\\ \hline
		text & string & Verbatim text content. \\ \hline
		idx & integer & The character offset of the token within the parent document. \\ \hline
	\end{tabular}
	\caption{Atributos de interés de la clase \textit{Tokenizer}}\label{tab:attspacy}
\end{table}


\subsection{textaCy}\label{ss:textacy}

Sobre la librería de spaCy se han desarrollado múltiples soluciones para distintos problemas en el ámbito del procesamiento de lenguaje natural. Uno de estos proyectos es textaCy\footnote{\url{https://spacy.io/universe/project/textacy}}. Se trata de una librería que cuenta con la capacidad de extraer los Information Items (véase la sección \ref{ss:resumen}), definidos como tuplas sujeto-verbo-objeto, de un texto. Basta con añadir la tarea de extracción de los InIts al pipeline de spaCy y, cuando se procese un texto, se obtendrán de forma automática estas tuplas. De esta manera, se puede generar una entrada para los correos electrónicos del corpus (que serían la salida del sistema) con la que entrenar el modelo construido.

\subsection{Tensorflow}\label{ss:tf}
Mientras que spaCy y textaCy son herramientas fundamentales durante el análisis y procesamiento del corpus, la librería de código abierto Tensorflow \citep{abadi2016tensorflow} es la piedra angular del desarrollo de la arquitectura transformer implementada. Posee una amplia cantidad de funcionalidades para trabajar con tensores de manera eficiente, está especializada en los modelos de aprendizaje automático y permite ejecutar Keras utilizando Tensorflow como base, la cual es una librería especialmente diseñada para implementar arquitecturas de deep learning y que facilita la construcción, entrenamiento y evaluación de las mismas.

\subsection{MongoDB}\label{ss:mongodb}
Como se justificará más adelante en la sección \ref{ss:almacen}, se necesita almacenar una gran cantidad de correos electrónicos con el fin de poder trabajar de forma eficiente con ellos. Para esta tarea se ha elegido MongoDB que es un sistema de base de datos NoSQL de código abierto y orientado a documentos.

En lugar de almacenar los datos en tablas, como se hace en las bases de datos relacionales, MongoDB almacena estructuras de datos BSON (una especificación similar a JSON) con un esquema dinámico, lo que facilita y agiliza la integración de datos en determinadas aplicaciones \citep{gyHorodi2015comparative}. Además, no se requieren recursos potentes para trabajar con ella y, gracias a la flexibilidad que ofrece el ser una base de datos NoSQL, se puede realizar fácilmente modificaciones en el modelo conceptual de la base de datos sin tener que preocuparse por los cambios problemáticos entre claves primarias y foráneas entre tablas. Además, cuenta con drivers oficiales para el lenguaje de programación Python con el que se desarrolla la solución.

\section{Enron corpus}\label{s:enron}
Para llevar a cabo este trabajo, se ha elegido el corpus conocido como Enron\footnote{\url{http://www-2.cs.cmu.edu/~enron/}}, dado que los correos electrónicos que contiene pertenecieron a la empresa con el mismo nombre. Precisamente se hicieron públicos tras una investigación legal llevada a cabo a esta compañía por parte de la Comisión Federal de Regulación de la Energía\footnote{\url{https://www.ferc.gov/}} de Estados Unidos.

Enron corpus contiene 517.401 correos electrónicos de 150 usuarios distintos. Además de la ventaja de la gran cantidad de elementos pertenecientes a este dataset, también ha sido elegido por encontrar diversos trabajos sobre este mismo conjunto de e-mails, como el llevado a cabo por \cite{klimt2004introducing}.
\section{Implementación de una arquitectura realizer}\label{s:realizer}

El planteamiento inicial de este estudio era el de desarrollar una arquitectura realizer que fuera capaz de redactar correos electrónicos de manera automática. Como se explicó detalladamente en la sección \ref{sss:realizer}, el primer problema que surge ante esta propuesta son las estructuras de datos ad hoc generadas en función del dominio del que se quiere producir los textos de lenguaje natural. Esto resulta un inconveniente debido a que, a diferencia de aplicaciones con un dominio definido como pueden ser aquellas cuyo propósito es el de generar informes meteorológicos, la redacción automática de correos electrónicos resulta extremadamente difícil de enmarcar dentro de uno o varios dominios específicos. Además, el corpus que se ha elegido para entrenar el modelo, no se restringe a un tipo de temática de e-mails, sino que pueden encontrarse mensajes que versan sobre una gran variedad de temáticas. Esto complica más la implementación porque, aunque se limitara a un número razonable de asuntos y este hecho no fuera desvelado, el modelo podría ser capaz de aprender el lenguaje específico y las características lingüísticas inherentes a los dominios (por supuesto con mayor dificultad que si se fuera consciente de esta ventaja). Sin embargo, al no contar con esta facilidad, no existen entidades concretas o propiedades comunes más allá de las que posee el lenguaje general.

Si se estudia en detalle la arquitectura realizer, se observa que el mayor problema de restricción de dominio se encuentra en la fase de determinación del contenido. Por supuesto que en el resto de fases también está presente en mayor o menor medida, por ejemplo en la estructuración del documento, pero no posee tanta dependencia como esta primera tarea localizada en el pipeline de las arquitecturas de generación de lenguaje natural. La razón es que resulta sumamente complicado conceptualizar en una estructura de datos concreta, cualquier posible intención que se pueda tener a la hora de transmitir cualquier información. Aún así, la solución traída desde el ámbito del resumen abstractivo de textos, en la que dichas intenciones se materializan en tuplas sujeto-verbo-objeto, solucionaba en gran medida la dificultad del paso de determinación de contenido. Es decir, esta fase se resolvía mediante la generación de Information Items que indican los temas que se desean tratar a lo largo de todo el correo electrónico. De hecho, la gran ventaja de llevar a cabo esta aproximación es que, no solo abordaba el problema de determinación del contenido que, en principio, parecía insalvable, sino que también ofrecía una solución para generar la entrada para el entrenamiento del sistema. Dicho de otro modo, el corpus elegido no proporciona más que los correos electrónicos, que en teoría son la salida final del sistema de generación de lenguaje natural, por lo que no se cuenta con una entrada. Con la solución de las tuplas sujeto-verbo-objeto, es posible conseguir una supuesta entrada desde la que podrían haberse redactado los mensajes. Esto da paso a la utilización de técnicas de aprendizaje supervisado y evita tener que producir una entrada escrita a mano. En resumidas cuentas, con los Information Items ``matábamos dos pájaros de un tiro'': se conseguía un método de generación automática de una entrada para los correos electrónicos ya generados y se solventaba el problema de no ser capaces de concretar el dominio de los e-mails del corpus de cara a diseñar estructuras de datos ad hoc necesarias para implementar la fase de determinación del contenido. Sin embargo, los Information Items introducen un problema sumamente complicado de abordar. Al tratarse de estructuras genéricas que pueden versar sobre cualquier temática, esto impide que puedan usarse técnicas que añadan algún tipo de información adicional en la generación del texto. El origen de este gran escollo es que las fases subsiguientes a la determinación de contenido son altamente dependientes de la salida de esta última. Cuando el dominio es concreto se puede razonar sobre el contenido (con técnicas como las ontologías) o contar con entidades preestablecidas para añadir información extra al texto producido, mientras que al no poder enmarcarse dentro de uno o varios dominios, actualmente no existe la forma de razonar sobre conocimiento general. En consecuencia, el sistema de generación de lenguaje natural se limitaría a recibir las tuplas sujeto-verbo-objeto y construir estructuras lingüísticas que incluyeran única y exclusivamente la información semántica que transmiten los InIts en cuestión. Es decir, simplemente se podrían producir, a lo sumo, tantas oraciones como Information Items se recibiera y la única tarea sería la de generar las escasas estructuras sintácticas que faltaran, así como ajustar algunas categorías morfológicas como el tiempo de los verbos y el género y número de las palabras. Esta técnica de Information Items solo ha sido empleada en los sistemas de resumen automático de textos, precisamente porque impide que en el resto de fases se pueda añadir información adicional más allá de estructuras sintácticas necesarias para la correcta construcción de las oraciones.

Ante este panorama en que las fases subsiguientes a la determinación de contenido poseen una dependencia excesivamente alta cuando se utilizan los Information Items, se descartó la posibilidad de desarrollar este sistema siguiendo el esquema de arquitectura realizer y se optó por generar la aplicación siguiendo el modelo transformer.
\section{Implementación de la arquitectura transformer}\label{s:transformer}

A diferencia de los problemas que se encuentran al tratar de construir una arquitectura realizer para un problema sin dominio específico como es la redacción automática de correos electrónicos, los transformers son capaces de afrontar la generación de lenguaje natural sin la necesidad de enmarcar los textos dentro de un ámbito concreto. De hecho, normalmente se suele acudir a esta arquitectura cuando se pretende hacer frente a problemas en los que podría requerirse un conocimiento general.

Como en el apartado \ref{sss:transformer}, se entró bastante en detalle en la estructura de la arquitectura, cada uno de sus módulos y la filosofía detrás de cada uno de ellos, esta sección se limitará a exponer las vicisitudes específicas del problema que compete a este trabajo, empezando por la entrada de la aplicación.

La entrada del modelo debe ser un conjunto de no más de seis Information Items construidos como tuplas sujeto-verbo-objeto. Sin embargo, esto no coincide del todo con la definición de la entrada del modelo transformer. Para resolverlo, en primer lugar se tomará la lista de todos los InIts y se tokenizarán por separado. La tarea de tokenización se lleva a cabo utilizando un tokenizador preentrenado\footnote{\url{https://www.tensorflow.org/text/api_docs/python/text/BertTokenizer}}.

Con una lista de las tuplas Inits tokenizadas, se concatenan como si constituyeran un solo tensor. Esto podría generar la preocupación de que es necesario antes homogeneizar los tensores asegurándose de que todos los sujetos poseen la misma longitud mediante la técnica de \textit{padding} (todos los tensores se adaptan a la longitud del mayor de ellos rellenando con ceros las últimas dimensiones), y lo mismo con los verbos y objetos. No obstante, además de ser una operación computacionalmente costosa, no es necesario para el correcto funcionamiento de nuestro modelo. El motivo por el que se pueden concatenar los InIts (sobra decir que siempre todos ellos deben seguir el orden sujeto-verbo-objeto, ya que esto sí podría dificultar el entrenamiento y confundir al modelo) y luego hacer padding es que, como se explicó en la sección \ref{sss:transformer}, existen tokens especiales de inicio y fin. Estos se añaden a cada sujeto, verbo y objeto por separado, quedando así todos los elementos claramente delimitados. Es decir, la red neuronal aprenderá a distinguir dónde acaba un elemento de la tupla InIt y dónde comienza el siguiente sin necesidad de incluir ceros entre ellos. Este hecho, no solo ahorra tiempo computacionalmente hablando, sino que también ahorra en memoria, pues los vectores tokenizados de InIts son notablemente más pequeños, y, al reducir el tamaño de los vectores de Information Items, disminuye la entrada de la red y, por ende, el número de parámetros a entrenar.

Tras construir la tokenización de los Information Items, se tokeniza, con el mismo tokenizador preentrenado, el correo electrónico ``resultante''. Así ya se cuenta con la entrada, el tensor concatenado de InIts, y la salida de la red, el tensor del cuerpo del mensaje tokenizado. De esta manera, se está en disposición de entrenar el modelo y evaluar los resultados obtenidos.
\section{Evaluación de los resultados}\label{s:resultados}

Para este modelo se ha utilizado el 80\% de los datos para el entrenamiento y el 20\% restante para el testeo, es decir, 185030 correos electrónicos con sus respectivos Information Items han entrenado el modelo y 46258 se han empleado para determinar la precisión del mismo.

A pesar de contar con un número tan elevado de correos electrónicos, en el entrenamiento se consiguió una precisión máxima de 0.4183 y en el conjunto de testeo de 0.2148. Hay que tener en cuenta que estos resultados se calculan como la probabilidad de predecir correctamente la próxima palabra dada una entrada y el texto generado en las iteraciones anteriores.

Una posible explicación que se le puede dar a estos resultados tan bajos es que, la entrada de los InIts no brinda suficiente contenido semántico como para generar todo el cuerpo del correo electrónico, es decir, el modelo transformer también se ve afectado por la dificultad que se encontraba en la arquitectura realizer de añadir información extra no incluida en los Information Items. A diferencia de otros casos de uso, como la traducción automática, en los que el modelo de atención sí puede hacer una correspondencia casi directa entre la entrada y la salida, esto no ocurre facilitándole como entrada estas tuplas sujeto-verbo-objeto, ya que la correspondencia de atención apenas se encuentra adecuadamente correspondida por contener la salida información semántica extra que resulta difícil generar sin dicho conocimiento.

Por lo tanto, aunque no debe descartarse la arquitectura transformer para este caso de uso, sí que conviene replantearse la definición de estas representaciones semánticas de entrada que quizás requieran ser más complejas para lograr unos resultados más satisfactorios.