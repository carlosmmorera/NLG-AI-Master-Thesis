\section{Generación de Lenguaje Natural}
%Intro de emails%
¿Y si fuera posible ahorrar todo este tiempo de escritura de correos electrónicos?

Para lograr este propósito es imprescindible profundizar en la rama de la Inteligencia Artificial conocida como \textit{Generación de Lenguaje Natural} (cuyas siglas son \textit{NLG} por su nombre en inglés \textit{Natural Language Generation}). Un buen ejemplo de aplicación de las técnicas de generación automática de textos son los 100.000 libros que Philip M. Parker puso a la venta en la plataforma \textit{Amazon.com} incluyendo títulos de temáticas tan variadas como \textit{El libro oficial del paciente sobre la estenosis espinal} (\textcolor{red}{citar}), \textit{Perspectivas mundiales 2009-2014 de los envases de 60 miligramos de Fromage Frais} (\textcolor{red}{citar}), \textit{Alfombras de baño} (\textcolor{red}{citar}), \textit{Juegos que miden 6 pies por 9 pies o menos en India} (\textcolor{red}{citar}) y \textit{Tesauro Quechua - Inglés} (\textcolor{red}{citar}).