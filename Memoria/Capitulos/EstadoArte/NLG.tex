\section{Generación de Lenguaje Natural}
%Intro de emails%
¿Y si fuera posible ahorrar todo este tiempo de escritura de correos electrónicos?

Para lograr este propósito es imprescindible profundizar en la rama de la Inteligencia Artificial conocida como \textit{Generación de Lenguaje Natural} (cuyas siglas son \textit{NLG} por su nombre en inglés \textit{Natural Language Generation}). Un buen ejemplo de aplicación de las técnicas de generación automática de textos son los 100.000 libros que Philip M. Parker puso a la venta en la plataforma \textit{Amazon.com} incluyendo títulos de temáticas tan variadas como \textit{El libro oficial del paciente sobre la estenosis espinal} \citep{parker2002official}, \textit{Perspectivas mundiales de 2009 a 2014 de los envases de 60 miligramos de Fromage Frais} \citep{parkerfromage},  \textit{Perspectivas de 2007 a 2012 de las tapetes de nudo, alfombras de baño y conjuntos que miden 6 pies por 9 pies o menos en la India} \citep{parkerrugs} y \textit{Tesauro Quechua - Inglés} \citep{parkerquechua}.

El algoritmo utilizado por Parker, se engloba dentro de los métodos de generación conocidos como \textit{text-to-text} (texto a texto en castellano), dado que toman como entrada textos ya existentes (normalmente escritos a mano y no generados automáticamente) y producen un nuevo texto coherente como salida. Otras aplicaciones de este tipo de métodos son la traducción automática de un idioma a otro \citep{hutchins2009introduction, oettinger2013automatic}, el resumen automático de textos \citep{mani2001automatic, nenkova2011automatic}, la simplificación de textos complejos, ya sea para hacerlos más accesibles para un público de lectores de bajo nivel de alfabetización \citep{siddharthan2014survey, bautista2011empirical} o niños \citep{macdonald2016summarising}, corrección automática de ortografía, gramática y texto \citep{kukich1992techniques, ng2014conll}, generación automática de revisiones de artículos científicos \citep{bartoli2016your}, generación de paráfrasis dada una frase de entrada \citep{bannard2005paraphrasing}, generación automática de preguntas con fines didácticos y educativos \citep{brown2005automatic} o generación automática de relatos dada una descripción conceptual de la historia deseada \citep{gervas2004story}.

Además de los métodos text-to-text, existen los llamados \textit{data-to-text} (datos a texto), en los cuales en lugar de recibir un texto como entrada, se genera el lenguaje a partir de datos. Estos pueden ser de todo tipo para dar lugar a informes o resúmenes como pueden ser de índole climatológica \citep{goldberg1994using, ramos2014linguistic}, financiera \citep{plachouras2016interacting}, ingenieril, como por ejemplo el trabajo desarrollado por \cite{yu2007choosing} para generar resúmenes de datos recopilados por sensores en turbinas de gas, sanitaria \citep{huske2003text, banaee2013towards}, como la investigación llevada a cabo por \cite{portet2009automatic} para obtener informes textuales a partir de datos de cuidados intensivos neonatales, o, incluso, deportivos \citep{theune2001data, chen2008learning}. Además de informes o resúmenes, también se utilizan los métodos \textit{data-to-text} para otros propósitos como la composición de discursos narrativos para relatos de varios personajes a partir de partidas de ajedrez \citep{gervas2014composing}, redacción de periódicos electrónicos a partir de datos de sensores \citep{molina2011generating}, generación de texto que aborda problemas medioambientales como el seguimiento de la fauna \citep{siddharthan2012blogging, ponnamperuma2013tag2blog}, la información medioambiental personalizada \citep{wanner2015getting} y la mejora del compromiso de los ciudadanos científicos a través de los comentarios generados \citep{van2016role} o producción de información interactiva sobre artefactos culturales \citep{stock2007adaptive}, entre otros.

Debido a que el objetivo de este trabajo se centra en la generación de correos electrónicos a partir del asunto, exploraremos en detalle las técnicas de Generación de Lenguaje Natural y, en especial, los métodos text-to-text. Para profundizar en los algoritmos y arquitecturas empleados ante los problemas de tipo data-to-text, conviene consultar la investigación llevada a cabo por \cite{gatt2018survey}, en la cual muestran el estado del arte de los trabajos realizados en este ámbito.

\subsection{¿Qué es la Generación de Lenguaje Natural?}
Dado que tanto los sistemas text-to-text como data-to-text y todas sus aplicaciones mencionadas anteriormente pertenecen a la rama de Generación de Lenguaje Natural, esta no debe definirse en función de la entrada del sistema, sino en la salida. Según \cite{biblia} la NLG es la conceptualización del ``campo de la inteligencia artificial y la lingüística computacional que se centra en los sistemas informáticos que son capaces de producir textos comprensibles en inglés u otra lengua humana. [...] Como área de investigación, la NLG presenta una perspectiva única ante problemas fundamentales de la inteligencia artificial, la ciencia cognitiva y la interacción. Estos incluyen cuestiones como por ejemplo cómo deben ser representados y cómo debe razonarse con la lingüística y el dominio del conocimiento, qué significa que un texto esté correctamente redactado y cómo es la mejor forma de comunicar información entre las computadoras y los usuarios.''