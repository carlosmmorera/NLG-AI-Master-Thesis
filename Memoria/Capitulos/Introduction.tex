\chapter{Introduction}
\label{cap:introduction}

Introduction to the subject area. This chapter contains the translation of Chapter \ref{cap:introduccion}.

\section{Motivation}
The recent development and consequent growth that smartphones have brought about overthe last few decades has brought with it innumerable paradigm alterations, both in terms oftechnology and in the social sphere. Precisely, the easy and increasingly simple access thatcame with the appearance of smartphones connected to a 3G network in the early 2000s brought tools, nowadays considered essential, such as videoconferencing or instant messaging, to an emerging and growing public. It was precisely this ``pioneer'' messaging system that ended up evolving and leading to the development of different applications that have helped to popularize real-time communication, with up to 100 billion messages being sent daily by 2020. Although this series of applications have the main characteristic of being closer, more informal and immediate -so that the need for immediacy and the urgency of the addressers. Although it is true that, nowadays, e-mail is generally and usually intended for more formal or institutional environments, as a prelude to a communication that will probably evolve towards other types of platforms or, more informally, for sending files, it continues to be a means of communication in regular use; in the case of some people, this use even extends to the daily environment, but with the clear and obvious disadvantage that it requires greater attention from the user.

It is obvious that there is a significant gap between instant messaging applications and e-mail and that this gap must be overcome in order to facilitate the experience and use of e-mail for its users; therefore, IRIS is proposed. The IRIS (Intelligent Response Instantly Sent) system is proposed as a virtual assistant whose objective is to improve the user experience in the use of e-mail, addressing the problem of the required dedication that it demands; therefore, the modus operandi of IRIS is simple. When using IRIS, the user postulates the idea they want to convey and the subject around which the body of the message should revolve and this is received by the assistant, which, by using natural language generation techniques, will try to express this idea in a text format in an extensive manner, to subsequently send the e-mail to its addressee.









