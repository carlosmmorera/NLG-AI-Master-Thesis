\chapter{Introduction}
\label{cap:introduction}

\chapterquote{Have you ever retired a human by mistake?}{Rachael - Blade Runner (1982)}

This chapter presents the motivation for the development of this work (see section \ref{motivation}), the objectives to be achieved (section \ref{obje}) and the structure of the document (section \ref{structure}).

\section{Motivation}\label{motivation}
The recent development and consequent growth that smartphones have brought about overthe last few decades has brought with it innumerable paradigm alterations, both in terms oftechnology and in the social sphere. Precisely, the easy and increasingly simple access thatcame with the appearance of smartphones connected to a 3G network in the early 2000s brought tools, nowadays considered essential, such as videoconferencing or instant messaging, to an emerging and growing public. It was precisely this ``pioneer'' messaging system that ended up evolving and leading to the development of different applications that have helped to popularize real-time communication, with up to 100 billion messages being sent daily by 2020. Although this series of applications have the main characteristic of being closer, more informal and immediate -so that the need for immediacy and the urgency of the addressers. Although it is true that, nowadays, e-mail is generally and usually intended for more formal or institutional environments, as a prelude to a communication that will probably evolve towards other types of platforms or, more informally, for sending files, it continues to be a means of communication in regular use; in the case of some people, this use even extends to the daily environment, but with the clear and obvious disadvantage that it requires greater attention from the user.

It is obvious that there is a significant gap between instant messaging applications and e-mail and that this gap must be overcome in order to facilitate the experience and use of e-mail for its users; therefore, IRIS is proposed. The IRIS (Intelligent Response Instantly Sent) system is proposed as a virtual assistant whose objective is to improve the user experience in the use of e-mail, addressing the problem of the required dedication that it demands; therefore, the modus operandi of IRIS is simple. When using IRIS, the user postulates the idea they want to convey and the subject around which the body of the message should revolve and this is received by the assistant, which, by using natural language generation techniques, will try to express this idea in a text format in an extensive manner, to subsequently send the e-mail to its addressee.

\section{Objectives}\label{obje}
The main objectives of this work are the study, implementation and evaluation of the different natural language generation (NLG) techniques that allow to automatically compose e-mails from small semantic representations. To achieve this, we will go into detail about the different architectures in the NLG field and we will delve into the alternatives to produce these semantic representations that can then give rise to complete messages. In particular, the so-called Information Items, used in the field of automatic text summarization, will be studied in order to implement these semantic representations and evaluate them with the two main architectures in the NLG field today: the transformer and realizer models.

The aim is to develop solutions able to, receiving as input the Information Items, compose an e-mail completely as the user would do it. To achieve this, a sufficiently large corpus of messages will be chosen to train the model adequately. This set of documents will be analyzed in order to know the characteristics and properties that define it, and tasks will be carried out to clean up the bodies of the e-mails and to process and store them. For this last phase, a suitable structure will be defined with an optimal storage system that will allow the information to be managed efficiently.

Once a storage system is in place and the data analysis and processing tasks have been carried out, the natural language generation architecture model that best fits the proposed problem will be implemented and trained with the corpus data. Finally, we intend to extract results that will show the performance of the developed system, and thus be able to evaluate the effectiveness of the proposed solution to the problem of automatic writing of e-mails with natural language generation techniques.

\section{Document structure}\label{structure}
This document presents the details of the work developed, showing the justification and motivation for the different decisions and steps that have been taken, and makes clear the different results, conclusions and reasoning that have been reached after analyzing everything implemented.

It begins, as it has been observed, with an introduction (chapter \ref{cap:introduction}) that tries to give an idea of the problem to which it is tried to give answer and the utility of the solution that is constructed. This motivation (section \ref{motivation}) is followed by the presentation of the objectives (section \ref{obje}) that will be covered throughout the project. These purposes will drive the decisions made throughout the work and will be the focus of attention at all times. The introduction ends with this section explaining the structure of the document.

Once the motivation and objectives of the study have been adequately introduced, a general review of the state of the art (chapter "state of the art") of the topics covered by the project is carried out. With this chapter, the reader will be able to adopt the necessary knowledge that forms the practical and theoretical basis for the presented development. Starting with the basics of e-mail (section \ref{s:email}), an overview is given of how this means of communication significantly influences today's society. In addition, the technical specifications necessary to understand how e-mails are formatted (the MIME protocol explained in section \ref{ss:mime}), how they are sent (the SMTP protocol is reflected in section \ref{ss:smtp}) and how they are received (POP and IMAP protocols are presented in sections \ref{ss:pop} and \ref{ss:imap}, respectively) are also given.

In chapter \ref{cap:estadoDeLaCuestion}, besides going into detail about e-mail, the necessary basics of natural language generation are provided (section \ref{s:nlg}): what this branch of artificial intelligence consists of and its most common applications (section \ref{ss:quenlg}), the main architectures most commonly used in NLG problems (section \ref{ss:arqnlg}) and a brief introduction to the field of abstractive text summarization (section \ref{ss:resumen}) from which techniques will be used to address the problem that concerns the project.
Then, when the knowledge required to tackle the problem is already available, all the work done is presented in chapter \ref{cap:descripcionTrabajo}. First, the technologies used in the development phase of the project are briefly introduced (section \ref{s:tech}): spaCy (section \ref{ss:spacy}), textaCy (section \ref{ss:textacy}), tensorflow (point \ref{ss:tf}) and MongoDB (section \ref{ss:mongodb}). It continues by reflecting all the steps involved in the preliminary data analysis of the chosen corpus (section \ref{s:analisis}) and ends by explaining the problems and details of the solution implementations using the realizer architecture (section \ref{s:realizer}), transformer (section \ref{s:transformer}) and the results obtained with the development (section \ref{s:resultados}).

Finally, in chapter \ref{cap:conclusions}, the conclusions that can be drawn from this study (section \ref{s:concle}) and the different open work paths left by this project on which it could be continued (section \ref{s:fute}) are brought to light.







