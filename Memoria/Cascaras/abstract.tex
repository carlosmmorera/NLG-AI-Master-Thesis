\chapter*{Abstract}

\section*{\tituloPortadaEngVal}

Nowadays, e-mail continues to be the dominant form of communication for both businesses and individual consumers. In fact, more than 319.6 billion e-mails are sent daily. This daily e-mail traffic translates into a great deal of time spent composing all the messages that are not sent automatically. But what if it were possible to avoid wasting all this e-mail writing time?

This problem can be solved by using natural language generation (NLG) techniques. For this reason, this paper evaluates the feasibility of implementing the main NLG architectures (realizer and transformer) to face the challenge of automatic e-mail writing, develops the proposed solution and evaluates the results obtained. The proposal presented includes as input a definition of a semantic representation unit used in automatic text summarization systems: the Information Items (InIts). The InIts will be the tool that materializes the idea that the user has to compose the e-mail and will be sent to the implementation of the realizer or transformer architecture developed to generate the body of the requested message.


\section*{Keywords}

\noindent e-mail, natural langugage generation, realizer, transformer, information items, automatic summarization, enron corpus, deep learning



