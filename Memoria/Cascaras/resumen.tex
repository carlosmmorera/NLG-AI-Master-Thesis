\chapter*{Resumen}

\section*{\tituloPortadaVal}

Hoy en día, el correo electrónico continúa siendo al forma de comunicación dominante tanto para las empresas como para los consumidores particulares. De hecho se envían más de 319.6 mil millones de e-mails diariamente. Este tráfico de correos diarios se traduce en una gran cantidad de tiempo invertido para la redacción de todos los mensajes que no se mandan de manera automática. Pero, ¿y si fuera posible ahorrar todo este tiempo de escritura de correos electrónicos?

Este problema puede ser resuelto mediante el uso de técnicas de generación de lenguaje natural (NLG). Por este motivo, en este trabajo se evalúa la viabilidad de implementar las principales arquitecturas de la NLG (realizer y transformer) para enfrentarse al reto de la redacción automática de correos electrónicos, se desarrolla la solución planteada y se valoran los resultados obtenidos. La propuesta presentada, incluye como entrada una definición de unidad de representación semántica utilizada en los sistemas de resumen automático de textos: los Information Items (InIts). Los InIts serán la herramienta que materialice la idea que el usuario tiene para redactar el correo electrónico y serán enviados a la implementación de la arquitectura realizer o transformer desarrollada para generar el cuerpo del mensaje solicitado.


\section*{Palabras clave}
   
\noindent correo electrónico, generación de lenguaje natural, realizer, transformer, information items, resumen automático de textos, corpus enron, deep learning

   


